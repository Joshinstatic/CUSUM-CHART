%%
%% This is file `thesis.tex',
%% generated with the docstrip utility.
%%
%% The original source files were:
%%
%% cnuthesis.dtx  (with options: `thesis')
%% 
%% This is a generated file.
%% 
%% Copyright (C) 2013 by Hoo Wan <imengyingchina@gmail.com>
%% 
%% This file may be distributed and/or modified under the
%% conditions of the LaTeX Project Public License, either version 1.3a
%% of this license or (at your option) any later version.
%% The latest version of this license is in:
%% 
%% http://www.latex-project.org/lppl.txt
%% 
%% and version 1.3a or later is part of all distributions of LaTeX
%% version 2004/10/01 or later.
%% 
%% To produce the documentation run the original source files ending with `.dtx'
%% through LaTeX.
%% 
%% Any Suggestions : Hoo Wan <imengyingchina@gmail.com>
%% Thanks Xue Ruini <xueruini@gmail.com> for the thuthesis class!
%% Thanks sofoot for the original SCNUThesis,NUDT paper class!
%% 

%1. 如果是研究生论文,常用的选项是:
% \documentclass[master,twoside,ttf]{cnuthesis}
%2. 如果是博士生论文,常用的选项是:
% \documentclass[doctor,twoside,ttf]{cnuthesis}
%3. 如果使用是Vista、Windows 7或者使用从Vista或Windows 7拷贝过来的字体,则需要再加一个Vista选项,如:
% \documentclass[master,twoside,ttf,vista]{cnuthesis}
%4. 建议使用OTF字体获得较好的页面显示效果
%   OTF字体从网上获得,各个系统名称统一,不用加vista选项
%   如果你下载的是最新的(1201)OTF英文字体,建议修改cnuthesis.cls,使用PS Std
%   \documentclass[doctor,twoside,otf]{cnuthesis}
%5. 如果想生成盲评,传递anon即可,仍需修改个人成果部分
% \documentclass[master,otf,anon]{cnuthesis}
%
\documentclass[master,twoside,ttf]{cnuthesis}
\usepackage{mycnu}
\usepackage{graphicx}
\usepackage{subfig}
\usepackage{listings}
\usepackage{float}
\usepackage{bm}


\begin{document}
\graphicspath{{figures/}}
\classification{G43}
\university{10028}
\confidentiality{无}
\serialno{2150502002}
\title{RACUF累积和控制图在监测长期病患生存时间的应用}
\displaytitle{首都师范大学硕士/博士学位论文\LaTeX{}模板示例文档}
\author{袁\ 钰}
\subject{应用统计}
\researchfield{应用统计}
\school{数学科学学院}
\supervisor{胡\ 涛}
\zhdate{2018 年 2 月 13 日}
% 插入摘要,制作封面
\pdfbookmark[1]{封面}{titlepage}
\ifisanon{}\else{\maketitle}\fi

\frontmatter

\input{data/abstract}
% 生成目录
{
\clearpage\pagenumbering{Roman}
\pdfbookmark[1]{目录}{tableofcontents}
\tableofcontents
}

\mainmatter
\pagestyle{mpage}
\input{data/chap01}
\input{data/chap02}
\input{data/chap03}
\input{data/chap04}
\input{data/chap05}
\input{data/chap06}

\clearpage
\bibliographystyle{bstutf8}
\bibliography{ref/refs}

\clearpage
\input{data/resume}

\clearpage
\input{data/ack}

\end{document}

%%
